\documentclass[a4paper,12pt]{report}
\renewcommand{\thesection}{\arabic{section}}
\renewcommand{\thesubsection}{\thesection.\arabic{subsection}}

% Packages nécessaires
\usepackage[french]{babel}
\usepackage[utf8]{inputenc}
\usepackage[T1]{fontenc}
\usepackage{graphicx}
\usepackage{xcolor}
\usepackage{geometry}
\usepackage{titlesec}
\usepackage{array}
\usepackage{longtable}
\usepackage{booktabs}
\usepackage{caption}
\usepackage{hyperref}
\usepackage[most]{tcolorbox}
\usepackage{tikz}

% Configuration des liens hypertextes
\hypersetup{
    colorlinks=true,
    linkcolor=naturacorpgreen,
    urlcolor=blue,
    citecolor=blue,
    pdftitle={Mise en Production - NaturaCorp},
    pdfauthor={Sellier Luka},
    pdfsubject={Documentation de mise en production},
    pdfkeywords={NaturaCorp, Déploiement, Production},
    pdfstartview={FitH},
    bookmarksnumbered=true,
    pdfpagemode=UseOutlines
}

% Mise en page
\geometry{margin=2.5cm}

% Couleurs personnalisées
\definecolor{naturacorpgreen}{RGB}{0,128,64}
\definecolor{lightgray}{HTML}{F2F2F2}

% Titres personnalisés
\titleformat{\section}
  {\normalfont\Large\bfseries\color{naturacorpgreen}}
  {\thesection}{1em}{}
\titleformat{\subsection}
  {\normalfont\large\bfseries}
  {\thesubsection}{1em}{}

% Colonnes personnalisées
\newcolumntype{L}[1]{>{\raggedright\arraybackslash}p{#1}}
\newcolumntype{C}[1]{>{\centering\arraybackslash}p{#1}}
\newcolumntype{R}[1]{>{\raggedleft\arraybackslash}p{#1}}
\renewcommand{\arraystretch}{1.3}

% En-tête et pied de page
\usepackage{fancyhdr}
\fancyhf{}
\fancyfoot[L]{Juin 2025}
\fancyfoot[C]{NaturaCorp - Mise en Production}
\fancyfoot[R]{\thepage}
\renewcommand{\footrulewidth}{0.4pt}
\pagestyle{fancy}

% Environnement pour notes
\newtcolorbox{notebox}[1][]{%
  breakable,
  colback=lightgray,
  colframe=naturacorpgreen,
  boxrule=1pt,
  arc=4pt,
  left=4mm,right=4mm,top=2mm,bottom=2mm,
  fonttitle=\bfseries,
  title=#1
}

\begin{document}

% Page de garde
\thispagestyle{empty}
\begin{center}
    % Logos
    \vspace*{0.5cm}
    \begin{minipage}{0.45\textwidth}
        \centering
        \includegraphics[width=5cm]{naturacorp.png}
        \vspace{0.3cm}
    \end{minipage}
    \hfill
    \begin{minipage}{0.45\textwidth}
        \centering
        \includegraphics[width=5cm]{esn.jpeg}
        \vspace{0.3cm}
    \end{minipage}
    \vspace{1.5cm}
    
    % Titre
    {\Huge\bfseries\color{naturacorpgreen} Mise en Production\par}
    \vspace{1.5cm}
    
    % Sous-titre
    {\LARGE\bfseries Documentation de déploiement\par}
    \vspace{2cm}
    
    % Auteur
    {\Large\bfseries Réalisé par :\par}
    \vspace{0.3cm}
    {\Large SELLIER Luka\par}
    \vspace{0.5cm}
    {\large Consultant Tech4Business\par}
    \vspace{0.3cm}
    {\large Bachelor 3 – Développement Web\par}
    \vspace{0.3cm}
    {\large École IRIS\par}
    \vspace{1.5cm}
    
    % Informations projet
    \begin{minipage}{0.8\textwidth}
        \centering
        \textbf{Projet de fin d'année}\\
        \vspace{0.2cm}
        Déploiement de la solution numérique\\
        pour une startup spécialisée dans les compléments alimentaires
    \end{minipage}
    \vspace{1.5cm}
    
    % Date
    {\large Mars 2025\par}
\vspace*{\fill}
\begin{center}
    \textbf{Livrable 4.1 : Mise en Production}
\end{center}
\end{center}

\newpage
\tableofcontents
\newpage

% Introduction
\section{Introduction}
\subsection{Objectif du document}
Ce document présente la procédure de mise en production de l'application NaturaCorp, en détaillant l'environnement cible, les prérequis, les étapes de déploiement, la gestion des risques et les vérifications nécessaires pour garantir le bon fonctionnement de la solution en production.

\subsection{Contexte}
Le projet NaturaCorp vise à déployer une application web sur un environnement de production sécurisé et évolutif, assurant la disponibilité et la performance du service pour ses utilisateurs finaux.

% ENVIRONNEMENT CIBLE
\section{Environnement cible}
\begin{description}
  \item[Type d’hébergement :] Serveur mutualisé Linux (Infomaniak)
  \item[Nom de domaine :] nc.1xprod.com
  \item[Système d’exploitation :] Linux
  \item[Serveur web :] Apache 2.4+
  \item[Base de données :] MySQL 5.7+
  \item[Accès :] SSH et panneau d’administration Infomaniak
\end{description}

% PRÉREQUIS TECHNIQUES
\section{Pré-requis techniques}
\subsection{Dépendances logicielles}
\begin{description}
  \item[PHP :] 8.2 ou supérieur, avec extensions : PDO, MySQL, GD, OpenSSL, Fileinfo, Mbstring, BCMath, Ctype, JSON, Tokenizer, XML
  \item[Node.js :] 18.x ou supérieur
  \item[NPM :] 9.x ou supérieur
  \item[Composer :] dernière version stable
  \item[Packages PHP :] Laravel Framework 12.0, Sanctum, DomPDF, Doctrine DBAL, Tinker, Intervention Image, Spatie Media Library
  \item[Packages JS :] @tailwindcss/forms, Alpine.js, Axios, Laravel Vite Plugin, Tailwind CSS, Vite
\end{description}

\subsection{Préparation de l’environnement}
\begin{itemize}
  \item Copier le fichier \texttt{.env.example} vers \texttt{.env} et configurer les variables d’environnement
  \item Générer la clé d’application : \texttt{php artisan key:generate}
  \item Installer les dépendances : \texttt{composer install --no-dev --optimize-autoloader}, \texttt{npm install}, \texttt{npm run build}
  \item Créer la base de données et configurer l’accès dans \texttt{.env}
  \item Exécuter les migrations : \texttt{php artisan migrate --force}
  \item Configurer les permissions sur \texttt{storage} et \texttt{bootstrap/cache}
\end{itemize}

% PROCÉDURE DE DÉPLOIEMENT
\section{Procédure de déploiement}
\subsection{Étapes principales}
\begin{enumerate}
  \item Compilation des assets front-end : \texttt{npm run build}
  \item Optimisation de Laravel : \texttt{php artisan config:cache}, \texttt{php artisan route:cache}, \texttt{php artisan view:cache}
  \item Synchronisation des fichiers vers le serveur (ex : via \texttt{rsync} ou SFTP)
  \item Installation des dépendances sur le serveur
  \item Exécution des migrations et création du lien symbolique de stockage
  \item Configuration des permissions
\end{enumerate}

\subsection{Script de déploiement}
Un script automatisé (\texttt{deploy.sh}) gère l’essentiel du processus de déploiement :
\begin{verbatim}
#!/bin/bash
# Compilation des assets
npm run build
# Optimisation de Laravel
php artisan config:cache
php artisan route:cache
php artisan view:cache
# Synchronisation des fichiers
rsync -avz --exclude='.git' --exclude='node_modules' --exclude='vendor' --exclude='.env' --exclude='storage/*' --exclude='bootstrap/cache/*' ./ utilisateur@serveur:/chemin/du/projet/
# Installation des dépendances et post-déploiement
ssh utilisateur@serveur "cd /chemin/du/projet && composer install --no-dev --optimize-autoloader && php artisan migrate --force && php artisan storage:link && chmod -R 775 storage bootstrap/cache && chown -R www-data:www-data storage bootstrap/cache"
\end{verbatim}

\subsection{Gestion de la base de données}
\begin{itemize}
  \item Création de la base via le panneau Infomaniak
  \item Configuration des accès dans \texttt{.env}
  \item Import ou migration des données : \texttt{php artisan migrate --force}
\end{itemize}

\subsection{Configuration post-déploiement}
\begin{itemize}
  \item Vérification des permissions sur les dossiers critiques
  \item Activation du SSL via le panneau d’administration
  \item Configuration du certificat (Let’s Encrypt)
  \item Redirection HTTP vers HTTPS
  \item Contrôle des accès utilisateurs
\end{itemize}

% GESTION DES RISQUES
\section{Gestion des risques}
\subsection{Risques potentiels}
\begin{longtable}{|L{4cm}|L{6cm}|L{4cm}|}
\hline
\textbf{Risque} & \textbf{Impact} & \textbf{Sévérité} \\
\hline
Panne serveur & Indisponibilité du service & Élevée \\
\hline
Erreur de configuration & Dysfonctionnements applicatifs & Moyenne \\
\hline
Problème de base de données & Perte de données ou inaccessibilité & Critique \\
\hline
Dépendance non satisfaite & Blocage du déploiement & Moyenne \\
\hline
Problème de DNS & Inaccessibilité temporaire & Élevée \\
\hline
\end{longtable}

\subsection{Plan de rollback}
\begin{itemize}
  \item Sauvegarde de la version précédente (fichiers et base)
  \item Restauration des fichiers et de la base
  \item Vérification des fonctionnalités principales
\end{itemize}

% VÉRIFICATIONS DE PRODUCTION
\section{Vérifications de production}
\subsection{Tests post-déploiement}
\begin{itemize}
  \item Vérification de l’accès à l’application (front et back-office)
  \item Contrôle des fonctionnalités principales (authentification, gestion des utilisateurs, commandes, etc.)
  \item Tests de performance (temps de réponse, charge serveur)
  \item Vérification SSL/TLS et sécurité
  \item Validation des rôles et permissions utilisateurs
\end{itemize}

\subsection{Surveillance et monitoring}
\begin{itemize}
  \item Surveillance manuelle des logs d’erreur via le panneau Infomaniak
  \item Vérification régulière de l’état du serveur et de l’application après déploiement
  \item Contrôle de la disponibilité du site en accédant régulièrement à l’URL de production
\end{itemize}

% DOCUMENTATION ANNEXE
\section{Documentation annexe}
\subsection{Commandes clés}
\begin{verbatim}
# Installation des dépendances
composer install --no-dev --optimize-autoloader
npm install
npm run build
# Migration de la base
php artisan migrate --force
# Optimisation
php artisan config:cache
php artisan route:cache
php artisan view:cache
# Permissions
chmod -R 775 storage bootstrap/cache
chown -R www-data:www-data storage bootstrap/cache
\end{verbatim}

\subsection{Configuration Apache}
\begin{verbatim}
<VirtualHost *:80>
    ServerName nc.1xprod.com
    DocumentRoot /var/www/html/public
    <Directory /var/www/html/public>
        AllowOverride All
        Require all granted
    </Directory>
</VirtualHost>
\end{verbatim}

\begin{notebox}[Note importante]
Ce document doit être mis à jour à chaque évolution technique ou organisationnelle du projet.
\end{notebox}

\end{document}
